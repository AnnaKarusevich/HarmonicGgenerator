%!TEX root = ../HarmonicGenerator.tex
% Тип документа
% \documentclass[a4paper,12pt]{extarticle}
\documentclass[a5paper,10pt]{article}
% \usepackage[pass]{geometry}
% Шрифты, кодировки, символьные таблицы, переносы
\usepackage{cmap}
\usepackage[T2A]{fontenc}
\usepackage[utf8]{inputenc}
\usepackage[russian]{babel}
% Это пакет -- хитрый пакет, он нужен но не нужен
\usepackage[mode=buildnew]{standalone}
\usepackage{gensymb}
\usepackage{textcomp}
\usepackage
	{
		% Дополнения Американского математического общества (AMS)
		amssymb,
		amsfonts,
		amsmath,
		amsthm,
		physics,
		% misccorr,
		% 
		% Графики и рисунки
		wrapfig,
		graphicx,
		subcaption,
		float,
		caption,
		color,
		booktabs,
		geometry,
		% 
		%
		% Интегралы и прочие обозначения
		ulem,
		esint,
		esdiff,
		% 
		% Колонтитулы
		fancyhdr,
	}  

\usepackage{xcolor}
\usepackage{hyperref}

 % Цвета для гиперссылок
\definecolor{linkcolor}{HTML}{000000} % цвет ссылок
\definecolor{urlcolor}{HTML}{799B03} % цвет гиперссылок
\hypersetup{pdfstartview=FitH,linkcolor=linkcolor,urlcolor=urlcolor, colorlinks=true}
\hypersetup{pageanchor=false}
% Увеличенный межстрочный интервал, французские пробелы
\linespread{1.3} 
\frenchspacing 



\newcommand\ct[1]{\text{\rmfamily\upshape #1}}
\newcommand*{\const}{\ct{const}}
\usepackage{array}
\usepackage{pstool}

\geometry		
	{
		left			=	1.5cm,
		right 			=	1.5cm,
		top 			=	1.5cm,
		bottom 			=	1.5cm,
		bindingoffset	=	0cm
	}

%%%%%%%%%%%%%%%%%%%%%%%%%%%%%%%%%%%%%%%%%%%%%%%%%%%%%%%%%%%%%%%%%%%%%%%%%%%%%%%
	%применим колонтитул к стилю страницы
% \pagestyle{fancy} 
% 	%очистим "шапку" страницы
% \fancyhead{} 
% 	%слева сверху на четных и справа на нечетных
% \fancyhead[R]{} 
% 	%справа сверху на четных и слева на нечетных
% % \fancyhead[L]{Отчёт по лабораторной работе №\labnumber} 
% 	%очистим "подвал" страницы
% % \fancyfoot{} 
% 	% номер страницы в нижнем колинтуле в центре
% \fancyfoot[C]{\thepage} 

%%%%%%%%%%%%%%%%%%%%%%%%%%%%%%%%%%%%%%%%%%%%%%%%%%%%%%%%%%%%%%%%%%%%%%%%%%%%%%%

\renewcommand{\contentsname}{Оглавление}
\usepackage{tocloft}
\usepackage{secdot}
\sectiondot{subsection}